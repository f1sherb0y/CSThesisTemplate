
{
\sectiontoconly{《浙江大学本科生毕业论文(设计)考核表》}
\fangsong
\begin{center}
    \xiaoer \bfseries
    毕业论文(设计)考~~核
\end{center}
\bfseries
\noindent
一、指导教师对毕业论文(设计)的评语:

{
\normalfont \fangsong
论文主要设计了一个基于 SiamFC 算法构建了脉冲神经网络算法, 并将其设计为硬件模块, 集成到基于RISC-V的SOC系统中。 论文基本结构完整,条例较为清楚, 但硬件的设计方案需要进一步完善,实验中需要进一步完善FPGA相关的结果。}

\vspace{2cm}


\begin{flushright}
    指导教师(签名) \underline{\hspace{3cm}} \hspace{1cm}

    \vspace{1cm}

    年 \hspace{1cm} 月 \hspace{1cm} 日
\end{flushright}


\noindent
二、答辩小组对毕业论文(设计)的答辩评语及总评成绩:

\vfill

\begin{center}
    \begin{tabular}{|p{1cm}|p{2cm}|p{2cm}|p{2cm}|p{3.2cm}|p{2cm}|}
        \hline
        \makecell{成绩     \\ 比例}                     & \makecell{文献综述/            \\ 中期报告 \\占(10\%)} & \makecell{开题报告 \\ 占(15\%)} & \makecell{外文翻译 \\ 占(5\%)} & \makecell{毕业论文(设计)\\质量及答辩 \\ 占(70\%)} & 总评成绩 \\
        \hline
        分值 &  &  &  &  & \\
        \hline
    \end{tabular}
\end{center}


\begin{flushright}
    答辩小组负责人(签名) \underline{\hspace{3cm}} \hspace{1cm}

    \vspace{1cm}

    年 \hspace{1cm} 月 \hspace{1cm} 日
\end{flushright}
\thispagestyle{empty}
}
