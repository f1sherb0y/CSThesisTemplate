%TC:group tabular 1 1
%TC:group table 1 1

\section{绪论}

图\ref{fig:zjulogo}是一个图形示例:浙江大学logo。

\begin{figure}[H]
    \includegraphics[width=0.2\textwidth]{figure/zjulogo.pdf}
    \caption[short]{浙江大学logo}
    \label{fig:zjulogo}
\end{figure}

\subsection{研究背景与意义}

这是一些引用\citep{exampleReference}。

\subsection{本文的结构}


\subsection{本研究的主要工作}



\clearpage
\section{背景知识和技术}

测试表格如表\ref{tab:test}所示。

\begin{table}[htbp]
    \wuhao\songti
    \centering
    \begin{tabular}{lllp{7cm}}
        \toprule
        第一列             & 第二列     & 第三列                    & 说明                           \\
        \midrule
        你好  &  &     & 说明 \\
        欢迎 &  &  & 说明              \\
        \bottomrule
    \end{tabular}
    \caption{神经元缓存模块信号}
    \label{tab:test}
\end{table}


\clearpage
\section{测试结果及其分析}

\subsection{实验环境}


\clearpage
\section{总结与展望}

\clearpage
\sectionnonum{参考文献}
\putbib

\clearpage
\sectionnonum{作者简历}


姓名:张三 ~ 性别:男 ~  民族:汉 ~ 出生年月:2002-1-1 ~  籍贯:xx省xx市

2014.09-2017.07 ~~  初中

2017.09-2020.07 ~~  高中

2020.09-2024.07 ~~  浙江大学攻读学士学位
